\documentclass[a4paper,titlepage,12pt]{article}
\usepackage[utf8]{inputenc}
\usepackage[T1]{fontenc}
\usepackage[margin=1in]{geometry}
\usepackage{parskip}
\usepackage{graphicx}
\usepackage{hyperref}
\usepackage{listings}
\usepackage{multirow}
\usepackage[usenames,dvipsnames]{color}
\hypersetup{
	colorlinks,
	pdfauthor=Delan Azabani,
	pdftitle=Programming Languages 200: parser assignment
}
\lstset{basicstyle=\ttfamily, basewidth=0.5em}

\title{Programming Languages 200:\\PL2014 parser assignment}
\date{October 12, 2014}
\author{Delan Azabani}

\pagenumbering{gobble}
\thispdfpagelabel0

\begin{document}

\maketitle

\pagenumbering{arabic}

\section{Overview}

This assignment submission includes a \texttt{Makefile} with the
following rules:

\begin{itemize}
	\item\texttt{PL2014\_check} compiles the main syntax checker,
	\item\texttt{clean} deletes all compiled and generated files,
	\item\texttt{test} executes a suite of 37 unit tests, and
	\item\texttt{report.pdf} typesets this report.
\end{itemize}

The default rule chosen upon invocation of \texttt{make(1)} without a
recipe is \texttt{PL2014\_check}.

I was fortunate enough to be able to develop the assignment without
once encountering a shift/reduce, reduce/reduce or any other kind of
conflict, so I have no idea what such an error from \texttt{yacc(1)}
looks like. Included with this submission is a \texttt{.git} directory
containing an uninteresting revision history over the course of my
work.

\section{The EBNF grammar}

\texttt{grammar.ebnf} contains the entire PL2014 grammar represented in
the Extended Backus--Naur Form as defined by ISO/IEC 14977:1996(E).
Each of the 33 production rules in the assignment specification's
syntax graphs has an equivalent with a matching name in the EBNF, in
addition to the following extra rules:

\begin{itemize}
	\item\texttt{constant\_declaration\_fragment} represents one of
	     the constituent comma separated equalities in a
	     \texttt{constant\_declaration},
	\item\texttt{variable\_declaration\_fragment} represents one of
	     the constituent comma separated declarations in a
	     \texttt{variable\_declaration},
	\item\texttt{compound\_statement\_sequence} represents a
	     semicolon-separated list of at least one statement,
	     excluding \texttt{BEGIN} and \texttt{END},
	\item\texttt{digit} represents any Arabic numeral, and
	\item\texttt{alpha} represents any Latin alphabetic character.
\end{itemize}

The former three rules are present to minimise redundancy and improve
clarity, while the remaining two are necessary because the EBNF
standard doesn't mandate the presence of any character classes.

In the given syntax graphs, I found the idea confusing that
\texttt{DECLARATION} and \texttt{END} in \texttt{declaration\_unit}
were separate tokens, but \texttt{END} and \texttt{IF}, \texttt{WHILE},
\texttt{DO} and \texttt{FOR} were not. I chose to standardise on
`separate words, separate tokens' to minimise dependence on any fixed
amount of whitespace between keywords.

It also appeared inconsistent that \texttt{if\_statement} would only
allow a single \texttt{statement} (even if said statement was a
\texttt{compound\_statement}) while the loop constructs allowed
multiple statements separated by semicolons. Because all of PL2014's
control structures frame statements firmly with keywords, thus keeping
them safe from ambiguity, I decided to upgrade \texttt{if\_statement}
to allow multiple statements.

\section{The lexical analyser}

\texttt{lexer.l} defines 17 keyword tokens, 17 symbol tokens, two
tokens with variety (\texttt{PL\_NUMBER} and \texttt{PL\_IDENT}), plus
a rule that ignores whitespace, and finally a rule that catches invalid
characters, terminating the analysis of input.

The \texttt{TOKV} function-like preprocessor macro is used to print out
the details of each encountered token with variety. \texttt{TOKE} is
similar, but does not print \texttt{yytext} for fixed symbols.

Assumptions made here include:

\begin{itemize}
	\item Keywords shall always consist of all capital letters,
	      thus yielding the convenient property of the keyword
	      and identifier namespaces failing to overlap, and
	\item Assignment and range delimitation symbols (\texttt{:=}
	      and \texttt{..} respectively) shall not contain spaces,
	      indicating the need for separate, unique tokens for these
	      symbols.
\end{itemize}

The order of definition of token rules mattered scarcely because
\texttt{lex(1)} is guaranteed to adhere to the maximal munch principle.
However, the fallback rule which was to catch any invalid character
must come after all other rules matching single characters.

\newpage

\section{The parser}

\texttt{parser.y} defines all 36 internal token identifiers used by
\texttt{lexer.l}. In addition, each of the 34 non-terminal symbols
outlined earlier in the EBNF grammar have been methodically translated
into \texttt{yacc(1)} rules with the same names.

The \texttt{NONT} function-like macro prints a message when a
non-terminal symbol has been completely parsed. Care was taken to
ensure that only left recursion was used in the translation from EBNF
to \texttt{yacc(1)} rules.

The following additional rules were required to represent optional
sequences:

\begin{itemize}
	\item\texttt{optional\_const\_and\_constant\_declaration},
	\item\texttt{optional\_var\_and\_variable\_declaration},
	\item\texttt{optional\_type\_declaration},
	\item\texttt{optional\_procedure\_interface},
	\item\texttt{optional\_function\_interface}, and
	\item\texttt{optional\_formal\_parameters}.
\end{itemize}

Optional sequences took the form:

\begin{lstlisting}
optional_thing :
               { NONT(optional_thing) }
               | thing
               { NONT(optional_thing) }
\end{lstlisting}

The following additional rules were required to represent repeated,
delimited sequences of at least one element:

\begin{itemize}
	\item\texttt{comma\_constant\_declaration\_fragments},
	\item\texttt{comma\_variable\_declaration\_fragments},
	\item\texttt{comma\_idents},
	\item\texttt{semicolon\_idents},
	\item\texttt{semicolon\_statements},
	\item\texttt{add\_subtract\_terms}, and
	\item\texttt{multiply\_divide\_id\_nums}.
\end{itemize}

Repeated sequences took the form:

\begin{lstlisting}
delimiter_things : thing
                 { NONT(optional_thing) }
                 | delimiter_things
                   delimiter
                   thing
                 { NONT(optional_thing) }
\end{lstlisting}

\end{document}
